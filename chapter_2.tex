\chapter{Estimation Theory}
\section{Introduction to Data Analysis}
\subsection{Data Analysis}
Data Analysis is the process of extracting information from data. Although data and information are often used interchangeably, they are not the same. Data is something that should cointain information, but it is not information itself. In order to extract information from data, we need to build a learning system.

An important thing to know is that when the learning system is \textit{good}, then by increasing the amount of data we have, the available information cannot decrease. In practice, most of the times we do not have a good system, so it can happen that the data fed to the system is misleading and the information we get from the analysis decreases.

The two main families of problems in data analysis are \textit{estimation} and \textit{classification}. The main difference is that in the estimation problem,
we have a set of data and we want to estimate a real value, while in the classi-
fication problem the output is contained in a finite set.

The input of the learning system can be anything (a vector, a matrix, a graph, a sequence,
etc.), but the output is always a real number or a finite set of values. In general,
we do not care about the dimension of the output, because we can always repeat
the problem as many times as the dimension of the output.

Another thing to make clear is that in statistical learning, we do not have a temporal correlation between the variables. When we say that two variables depend on each other, we just mean that they are correlated, without implying the causality.

% qualcosa sul fatto che esiste model based analysis e supervised parametric e non parametric

\section{Bayes Estimation}
Assume we have a random variable $Y$ and we want to approximate it with a single deterministic value. To do that, we need to find the number that, on average, is the closest to the values of the random variable. In other words, we want to find the number $z^\ast$ that minimizes the expected value of the squared \textit{distance} between $Y$ and $z$:
\[
    z^\ast \colon \arg \min_{z \in \mathbb{R}} \E{}{(Y-z)^2}
\]
The quantity we want to minimize is called \textit{mean squared error} (MSE) and in this case is a function of the value $z$. We can find the minimum of this function by computing its derivative and setting it to zero:

\section{Maximum Likelihood Estimation}
In the bayesian setting the parameter was a random variable. Now let us consider the case in which the parameter is a deterministic quantity, but it is unknown.

Assume that we have a random variable $X \in \mathbb{R}^d$ that is generated by an unknown distribution that depends on a parameter $\theta \in \mathbb{R}$.

The parameter can be considered as the \textit{truth} that generates the data and we want to find an estimator $\hat{\theta}$ for $\theta$ that is a function of the data $X$.
\[
    \theta \to X \to \boxed{ML Estimation} \to \hat{\theta} = g(X)
\]

As \textbf{error metric} we can use the \textit{mean squared error}:
\[
    h(\theta) = \E{}{(\theta - \hat{\theta})^2} = \E{}{(\theta - g(X))^2}
\]
In the bayesian setting the error metric was the following:
\[
    h(Y,X) = \E{}{(Y-g(X))^2}
\]
The difference between the two error metrics is that in the first case, the expression is a function of $\theta$, in the second case the expression is a number, because the expectation of a random variable is a scalar value by

If we want to minimize the error metric $h(\theta)$, we could compute a degenerate solution that is $g(X) = \theta$. In this case the error will be zero, but we would reach a contradiction, because that would mean that we already know $\theta$, which is not true because it is our parameter.

The maximum likelihood strategy starts from the assumption that the estimator $g(X)$ is unbiased (condition of \textit{unbiasedness}).

\begin{definition}
    The likelihood function is a function of the parameter $\theta$ that is defined as:
    \[
        \ell(x | \theta) = \mathbb{P}(X = x | \theta)
    \]
    where $x$ is a realization of the random variable $X$.
\end{definition}

\begin{theorem}
    In order to find the \textbf{maximum likelihood estimator}, we need to find the value of $\theta$ that maximizes the likelihood function.
    \[
        \hat{\theta}_{ML} = \arg\max_{\theta} \ell(x; \theta)
    \]
\end{theorem}

Assume that we know the shape of the likelihood functon. In order to find a point of maximum we need to solve the following equations:
\begin{align*}
    \frac{\partial \ell(x;\theta)}{\partial \theta} = 0 \\
    \frac{\partial \ell(x;\theta)}{\partial^2 \theta} \geq 0
\end{align*}
The equations have a closed form solution only if the likelihood function is derivable. If the likelihood function is not derivable, we need to use a gradient ascent algorithm. However, the latter is not garanteed to always work, because in order for the solution of the algorithm to converge to the point of true maximum we need jumps that are proportional to the derivative, decreasing the learning rate.

\paragraph*{High dimensional parameters}
The \textit{maximum likelihood strategy} works also if the parameter is a vector. If $\vec{\theta} \in \mathbb{R}^m$ we will find that:
\[
    \hat{\theta} = \arg\max_{\theta }  \ell(\vec{x}; \theta)
\]
However, we should be careful while increasing the dimension of the parameter, because of \textit{curse of dimensionality problem}; we have an exponential dependence between the number of dimensions and the number of elements in the dataset.

We may have various shapes of the likelihood function:
\begin{itemize}
    \item there may be a \textit{subgradiant shape}
    \item there may be a situation in which we have local maxima and minima
\end{itemize}
In that case we may use a convex or concave optimization strategy\dots
\paragraph*{Properties of the Maximum Likelihood Estimator}
Maximum likelihood estimators are also called Minimum-Variance Unbiased (\textit{MVU}) estimators. This is because, given independent data, they are:
\begin{enumerate}
    \item asymptotically unbiased
    \item efficient, meaning that they find the estimate with minimum variance. The minimum value of variance is bounded by the Cramer-Rao lower bound, that is defined the Fischer Information matrix.
    \item have \textbf{large-sample optimality}, meaning that it is optimal if we have many data, independently of the particular problem.
\end{enumerate}

\section{Exercises}
\subsection{Maximum Likelihood}
\subsection{Mean estimation on homogeneous data}
\begin{exercise}
    Compute the maximum likelihood estimator for the mean of a gaussian random variable with known variance.
\end{exercise}
Given data $X \in \mathbb{R}$ and variance $\sigma^2$, the likelihood function is:
\[
    \ell(x;\theta) = \frac{1}{\sqrt{2\pi\sigma^2}} \exp{-\frac{(x-\theta)^2}{\sigma^2}}
\]
In this special case, the likelihood function for a fixed $X$ has the same shape of that we would have for a fixed value of the parameter, that is the normal curve. It is intuitive, by looking at the curve that the point of maximum is the mean of the distribution, that is the parameter $\mu$.

Let us now consider the case when $X$ becomes a vector. We have that $X \in \mathbb{R}^N$, where
$X = \left[x_1, \dots, x_N\right]^T$ are all \textit{iid} guassian random variables; and $\sigma^2$ is known. We need to compute the joint PDF of all of the variables.
\begin{align*}
    \ell(\vec{x};\theta) = \prod_{i=1}^{N} \ell(x;\theta) = \prod_{i=1}^{N} \left(\frac{1}{\sqrt{2\pi\sigma^2}} \exp{-\frac{(x_i - \theta)^2}{2\sigma^2}}\right) \\
    = \left(\frac{1}{\sqrt{2\pi\sigma^2}}\right)^n \exp{-\frac{\sum_{i=1}^{N} (x_i - \theta)^2}{2\sigma^2}}
\end{align*}
The constant is irrelevant to find the point of maximum; so we have:
\[
    \exp{-\frac{\sum_{i=1}^{N} (x_i - \theta)^2}{2\sigma^2}}
\]
But the maximum value of the exponential function is found for the minimum value of its exponent, because this function is monotonically decreasing. So we have that the argmax of the likelihood function in this case is equivalent of the argmin of its exponent:
\[
    \arg\max_{\theta} - \sum_{i = 1}^{n} \frac{(x_i - \theta)^2}{2 \sigma^2} \Leftrightarrow \arg\min_{\theta} \sum_{i = 1}^{n} \frac{(x_i - \theta)^2}{2 \sigma^2}
\]
Now we need to compute the derivative wrt $\theta$ and equate to zero:
\[
    \frac{\partial \sum(x_i-\theta)^2}{\partial \theta} = 0 \Leftrightarrow 2N\sum_{i=1}^N(\theta - x_i) = 0 \Leftrightarrow \sum_{i=1}^N \theta = \sum_{i=1}^N x_i \]

\[
    N\theta = \sum_{i=1}^N x_i \Rightarrow \hat{\theta}_{MLE}  = \frac{\sum_{i=1}^N x_i}{N}
\]
To check if the obtained point is a point of minimum we need to compute the second derivative:
\[
    \frac{\partial^2 \sum_i(x_i-\theta)^2}{\partial \theta^2} = 0 \Leftrightarrow 2N > 0
\]

Note that this is a special case where the estimator we found is optimal and is the same of the arithmetic mean.

\subsection{Mean estimation on heterogeneous data}

Let's consider the following setting: we have $N$ independent random variables $X_i \in \mathbb{R}$ which represent our data and a scalar parameter $\theta \in \mathbb{R}$ representing the \textit{mean} that we want to estimate.

We have an \textit{heterogeneous} set of data; half of our data has variance $\sigma_1^2$ and the other half has variance $\sigma_2^2$. This can represent a situation where we have two different sensors that measure the same quantity but with different precision, which is represented by the variance of the data.

Our data is the following:
\begin{align*}
     & X_i \sim N(\theta, \sigma_1^2) \qquad \text{for } i = 1 \dots \frac{N}{2}     \\
     & X_i \sim N(\theta, \sigma_2^2) \qquad \text{for } i = \frac{N}{2} + 1 \dots N
\end{align*}

Note that we can also write $X_i$ as $X_i = \theta + W_i$, where each noise term is unbiased (zero mean) and independent from each other and has variance $\sigma_1^2$ or $\sigma_2^2$.

Now we want to answer the following questions:
\begin{itemize}
    \item \textit{Should I throw out the data with the higher variance?}
    \item \textit{Can I use the arithmetic mean or should I use a different estimator?}
\end{itemize}

We know only one method to find the best estimator: the Maximum Likelihood Estimator. We can use it to find the best estimator for this problem.

Let us compute the likelihood function first. Since the data is independent, we can write the likelihood as the product of the likelihoods of each component of the data:
\[
    \ell(x;\theta) = \prod_{i=1}^{\frac{N}{2}} \frac{1}{\sqrt{2\pi\sigma_1^2}} \exp{-\frac{(x_i - \theta)^2}{2\sigma_1^2}} \prod_{i=\frac{N}{2}+1}^{N} \frac{1}{\sqrt{2\pi\sigma_2^2}} \exp{-\frac{(x_i - \theta)^2}{2\sigma_2^2}}
\]
We can simplify the expression by moving the constants out of the product:
\[
    \ell(x;\theta) = \frac{1}{\left(\sqrt{2\pi\sigma_1^2}\right)^{\frac{N}{2}}}  \prod_{i=1}^{\frac{N}{2}} \exp{-\frac{(x_i - \theta)^2}{2\sigma_1^2}} \prod_{i=\frac{N}{2}+1}^{N} \frac{1}{\sqrt{2\pi\sigma_2^2}} \exp{-\frac{(x_i - \theta)^2}{2\sigma_2^2}}
\]
Then we further simplify the expression:
\[
    \ell(x;\theta) = \frac{1}{\left(2\pi\sqrt{\sigma_1^2\sigma_2^2}\right)^{\frac{N}{2}}} \exp{ - \left[\sum_{i=1}^{\frac{N}{2}} \frac{(x_i - \theta)^2}{2\sigma_1^2} + \sum_{i=\frac{N}{2} + 1}^{N} \frac{(x_i - \theta)^2}{2\sigma_2^2}\right]}
\]
We can remove the constants since they do not depend on $\theta$. Then, since the exponential is a monotonic function, we can find the argmax by maximizing the exponent only. Moreover, since solving a maximum problem is equivalent to solving a minimum problem, by changing sign, we have that:
\[
    \hat{\theta} = \arg\max \ell(x; \theta) \iff \hat{\theta} = \arg\min \left[\sum_{i=1}^{\frac{N}{2}} \frac{(x_i - \theta)^2}{2\sigma_1^2} + \sum_{i=\frac{N}{2} + 1}^{N} \frac{(x_i - \theta)^2}{2\sigma_2^2}\right]
\]
Now to solve the problem we can take the derivative of the expression and set it to zero:
\begin{align*}
     & \frac{d}{d\theta} \left[\sum_{i=1}^{\frac{N}{2}} \frac{(x_i - \theta)^2}{2\sigma_1^2} + \sum_{i=\frac{N}{2} + 1}^{N} \frac{(x_i - \theta)^2}{2\sigma_2^2}\right] = 0                      \\
     & \Leftrightarrow \frac{1}{\sigma_1^2} \sum_{i=1}^{\frac{N}{2}} \frac{d}{d\theta}(x_i - \theta)^2 + \frac{1}{\sigma_2^2} \sum_{i=\frac{N}{2} + 1}^{N} \frac{d}{d\theta}(x_i - \theta)^2 = 0 \\
     & \Leftrightarrow \frac{1}{\sigma_1^2} \sum_{i=1}^{\frac{N}{2}} 2(\theta -x_i) + \frac{1}{\sigma_2^2} \sum_{i=\frac{N}{2} + 1}^{N} 2(\theta - x_i) = 0
\end{align*}
Now we split the sums:
\[
    \frac{1}{\sigma_1^2} \sum_{i=1}^{\frac{N}{2}} \theta + \frac{1}{\sigma_2^2} \sum_{i=\frac{N}{2} + 1}^{N} \theta = \frac{1}{\sigma_1^2} \sum_{i=1}^{\frac{N}{2}} x_i + \frac{1}{\sigma_2^2} \sum_{i=1}^{\frac{N}{2}} x_i
\]
\[
    \frac{1}{\sigma_1^2} \frac{N}{2} \theta + \frac{1}{\sigma_2^2} \frac{N}{2} \theta = \frac{1}{\sigma_1^2} \sum_{i=1}^{\frac{N}{2}} x_i + \frac{1}{\sigma_2^2} \sum_{i=1}^{\frac{N}{2}} x_i
\]
Now we divide both sides by $\frac{N}{2}$ and we obtain arithmetic averages in the second member. Let us call $\overline{x_1}$ the arithmetic average of the first half of the data and $\overline{x_2}$ the arithmetic average of the second half of the data. Then we have:
\[
    \frac{1}{\sigma_1^2} \theta + \frac{1}{\sigma_2^2} \theta = \frac{1}{\sigma_1^2} \overline{x}_1 + \frac{1}{\sigma_2^2} \overline{x}_2
\]
By factoring $\theta$ we obtain:
\[
    \hat{\theta} \left(\frac{\sigma_1^2 + \sigma_2^2}{\sigma_1^2\sigma_2^2}\right) = \frac{\sigma_2^2\overline{x}_1 + \sigma_1^2\overline{x}_2}{\sigma_1^2 \sigma_2^2}
\]
So we solve for theta:
\[
    \hat{\theta}_{ML} = \left(\frac{\sigma_1^2\sigma_2^2}{\sigma_1^2 + \sigma_2^2}\right) \frac{\sigma_2^2\overline{x}_1 + \sigma_1^2\overline{x}_2}{\sigma_1^2 \sigma_2^2} = \frac{\sigma_2^2\overline{x}_1 + \sigma_1^2\overline{x}_2}{\sigma_1^2 + \sigma_2^2} = \frac{\sigma_2^2}{\sigma_1^2 + \sigma_2^2} \overline{x}_1 + \frac{\sigma_1^2}{\sigma_1^2 + \sigma_2^2} \overline{x}_2
\]
\paragraph*{Asmpytotic behaviour}
We can see that it is a weighted average of the two arithmetic averages of the two halves of the data.

We can write the estimator as:
\[
    \hat{\theta} = p \overline{x}_1 + (1 - p) \overline{x}_2
\]
where
\[
    p = \frac{\sigma_2^2}{\sigma_1^2 + \sigma_2^2}
\]
Since the two weights sum to one and positive and smaller than one, we say that the estimator is a \textbf{convex combination} of the two arithmetic averages. This combination has the property that it always lies between the two averages.

Let us consider the case when $\sigma_1^2 = \sigma_2^2$. We have that $p = \frac{1}{2}$ and the estimator is the arithmetic average of the two halves of the data, which is equal to the arithmetic average of all the data.
\[
    \hat{\theta}_{avg} = \frac{1}{2} \overline{x}_1 + \frac{1}{2} \overline{x}_2 = \frac{1}{N} \sum_{i=1}^{N} x_i
\]
Let us consider the case when $\sigma_2^2 \gg \sigma_1^2$. We have that $p \approx 1$ and the estimator is almost equal to the arithmetic average of the first half of the data. This is also true for the opposite case.

From this analysis, we can understand that if the variance of one half of the data is much larger than the other half of the data, then we \textit{could} discard the data with the larger variance, \textbf{however} we would lose information. The solution is to give more weight to the data with the smaller variance.

\paragraph*{Unbiasedness}
Now we can verify that the maximum likelihood estimator $\theta_{ML}$, the plain arithmetic average $\theta_{avg}$ and the estimators that uses only one half of the data $\theta_{1}$ and $\theta_{1}$ are all unbiased.
\[
    \E{}{\hat \theta_1} = \E{}{\frac{1}{\frac{N}{2}} \sum_{i=1}^{\frac{N}{2}} X_i} = \frac{1}{\frac{N}{2}} \sum_{i=1}^{\frac{N}{2}} \E{}{X_i} = \frac{1}{\frac{N}{2}} \sum_{i=1}^{\frac{N}{2}} \theta = \theta
\]

\[
    \E{}{\hat \theta_{avg}} = \E{}{\frac{1}{N} \sum_{i=1}^{N} X_i} = \frac{1}{N} \sum_{i=1}^{N} \E{}{X_i} = \frac{1}{N} \sum_{i=1}^{N} \theta = \theta
\]

\[
    \E{}{\hat \theta_{ML}}=\E{}{p\bar X_1+(1-p)\bar X_2}=p\E{}{\bar X_1}+(1-p) E{\bar X_2}=p\theta+(1-p)\theta=\theta
\]

\paragraph*{Comparison}
Since all of the estimators are unbiased, in order to compare the different estimators, we need to compute the variance of each estimator. The one with the smallest variance will be the most efficient. For the two splits of data we have:
\[
    \var{\hat \theta_1} = \var{\frac{2}{N} \sum_{i=1}^{\frac{N}{2}} X_i} = \frac{2}{N}\sigma_1^2
\]
\[
    \var{\hat \theta_2} = \var{\frac{2}{N} \sum_{i=\frac{N}{2} + 1}^{N} X_i} = \frac{2}{N}\sigma_2^2
\]
For the arithmetic average we have:
\[
    \var{\hat \theta_{avg}}=Var\left[\frac 12\hat X_1+\frac 12 \hat X_2\right]=\frac 14\frac{2\sigma_1^2}N+\frac 14 \frac{2\sigma_2^2}N=\frac{\sigma_1^2}{2N}+\frac{\sigma_2^2}{2N}= \frac{(\sigma_1^2+\sigma^2_2)}{2N}
\]
For the maximum likelihood estimator we have:
\begin{align*}
    \var{\hat{\theta}_{ML}} = Var\left[\frac{\sigma_2^2}{\sigma_1^2+\sigma_2^2}\hat X_1+\frac{\sigma_1^2}{\sigma_1^2+\sigma_2^2}\hat X_2\right]=                                            \\
    = \left(\frac{\sigma_2^2}{\sigma_1^2+\sigma_2^2}\right)^2 \var{\hat X_1}+\left(\frac{\sigma_1^2}{\sigma_1^2+\sigma_2^2}\right)^2 \var{\hat X_2} =                                       \\
    = \frac{2}{N} \left(\frac{\sigma_2^2}{\sigma_1^2+\sigma_2^2}\right)^2 \sigma_1^2 + \frac{2}{N} \left(\frac{\sigma_1^2}{\sigma_1^2+\sigma_2^2}\right)^2 \sigma_2^2 =                     \\
    = \frac{2}{N} \frac{\sigma_1^2\sigma_2^4+\sigma_1^4\sigma_2^2}{(\sigma_1^2+\sigma_2^2)^2} = \frac{2}{N} \frac{\sigma_1^2\sigma_2^2(\sigma_1^2+\sigma_2^2)}{(\sigma_1^2+\sigma_2^2)^2} = \\
    = \frac{2}{N} \frac{\sigma_1^2\sigma_2^2}{\sigma_1^2+\sigma_2^2}
\end{align*}

Now we can to compare the different estimators. The best estimator is the one with the lowest variance. We're going to prove that the maximum likelihood estimator is the best of the other estimators. To simplify the notation, we're going to call $\sigma_1^2 = s_1$ and $\sigma_2^2 = s_2$.

Let us compare first the maximum likelihood estimator with the estimator that uses only the first half of the data. We have that:
\begin{align*}
    Var[\hat \theta _{ML}]\overset ?< Var[\hat \theta_1]=         \\
    = \frac {2}{ N} \frac{{s_1}s_2}{(s_1+s_2)}<\frac{{s_1}}{ N/2} \\
    = \frac{s_2}{(s_1+s_2)}<1
\end{align*}
which is true because $s_2<s_1+s_2$. Now let us compare the maximum likelihood estimator with the arithmetic average:
\begin{align*}
    Var[\hat \theta _{ML}]\overset ?< Var[\hat \theta_{avg}]     \\
    \frac 2{ N} \frac{s_1s_2}{(s_1+s_2)}< \frac {(s_1+s_2)}{2 N} \\
    4 {s_1s_2}< (s_1+s_2)^2                                      \\
    4 {s_1s_2}< s_1^2+s_2^2+2s_1s_2                              \\
    s_1^2+s_2^2-2s_1s_2>0                                        \\
    (s_1-s_2)^2>0
\end{align*}
Finally, let us compare the arithmetic average with the estimator that uses only the first half of the data:
\begin{align*}
    Var[\hat \theta _{ave}]\overset ?< Var[\hat \theta_{1}] \\
    \frac {(s_1+s_2)}{2 N}<\frac 2{ N}s_1                   \\
    \frac {(s_1+s_2)}{4}< s_1                               \\
    {(s_1+s_2)}< 4{s_1}                                     \\
    -3s_1+s_2<0                                             \\
    s_2<3s_1
\end{align*}

In this case, the inequality is not always true. We can see that if the variance of the second half of the data is less than three times the variance of the first half of the data, then the arithmetic average is better than the estimator that uses only the first half of the data. In this case there is \textbf{hard threshold} that tells us when to use the arithmetic average and when to use the estimator that uses only the first half of the data. In the other cases, the maximum likelihood estimator is always better than the other two estimators.

\begin{remark}
    Observe that all of the variances are inversely proportional to the variance of the data divided by the number of samples. This seems to be a common trend but it is not always a general rule, because it depends on the strategy we're implementing.

    If we're implementing an optimal strategy, then if we increase the dataset size then we cannot go worse than before. This is because, even if we're adding irrelevant data, we're adding information that it is already included in the data.

    If however we're using another suboptimal strategy, we cannot assume that increasing the dataset size our estimator improves its performance (i.e. decreases its variance).

    In conclusion, if (1) we're using an optimal strategy and (2) we're adding relevant data, we can improve the performance of our estimator.
\end{remark}
\subsection{MMSE}
\begin{exercise}
    We have a random variable $Y \sim N(0, \sigma^2_y)$ and data
    \[
        X_i = Y_i + W_i \qquad i = 1, \dots, n
    \]
    where variables $W_i$ are \textit{i.i.d} and also independent with respect to $Y$ and distributed according to $W \sim N(0, \sigma_w^2)$.

    Compute the \textit{minimum mean square error estimator}.
\end{exercise}

The MMSE is the posterior mean of $Y$ given $X$:
\[
    \hat{Y} = \mathbb{E}\left[ Y | X\right]
\]
To compute the posterior mean, we need to compute the posterior distribution of $Y$ given $X$. We can do that either by definition or by applying Bayes' Theorem:
\[
    f(y|x) = \frac{\pi(y) \ell(x|y)}{p(x)}
\]


We can infer that:
\[
    f(x_i|y) = \ell(x_i|y) = \frac{1}{\sqrt{2\pi \sigma^2_w}} e^{\frac{-(x-y)^2}{2\sigma_w^2}}
\]
and consequently
\[
    \ell(x|y) = \prod_{i = 1}^{N} \ell(x_i, y)
\]
While we can say that given $Y=y$ then $X_i$ is distributed according to $W_i$ that is shifted by an amount of $y$, because $X_i = Y_i + W_i$ and $f(w|y) = f(w)$, we cannot compute $f(y|x)$ observing that $Y_i = X_i - W_i$ because $X_i$ and $W_i$ are not independent on each other but they are dependent because $X_i = Y_i + W_i$. Thus we need to apply Bayes' Theorem to find out $\ell(y|x)$.

Observe that since we are computing $x$ \textit{given} $y$, then we're actually trying to understand the \textbf{generative mechanism} that produces the data $x$ given the true value $y$. Suppose for example that we're sampling measurements $x$ of the temperature in a room. We're actually measuring the true value of the temperature and some noise $w$ added to it. We can try to estimate the generative mechanism to understand what is the distribution of temperature given the true value.

Recall that, in the Bayesian setting, $f(y|x)$ is called \textit{posterior} function and $\pi(y)$ is called the \textit{prior} function. The posterior function does not tell us what is the generative mechanism of the data but it estimates how $y$ is hidden from the data. Also, we will observe that the \textbf{MMSE} estimator is obtained by combination of likelihood and prior information.

In order to compute $f(y|x)$ we can apply the Bayesian Theorem:
\[
    f(y|x) = \frac{\pi(y) f(x|y)}{p(x)}
\]
Observe that since $f(y|x)$ is a probability density function with respect to $y$, the term $p(x)$ must be a constant with respect to $y$. We can express $p(x)$ as the joint or \textit{marginal} distribution of $x$ and $y$ because
\[
    p(x) = \int \pi(y^{\prime}) \ell(x|y^{\prime} ) dy
\]
Because if we apply to the previous expression the integral to both members, we obtain:
\[
    \int f(y|x) = \frac{\int \pi(y) \ell(x|y) dy}{p(x)} = 1 ???
\]

So, given that $p(x)$ is a constant with respect to $y$, we arrive at the fundamental proposition that:
\[
    f(y|x) \propto \pi(y) \ell(x|y)
\]
It is propotional because I divide by $p(x)$ that is a factor that is completely determined by $y$. If $p(x)$ is not a constant given $y$ then $f(y|x)$ is not a probability density function.

Some remarks:
\begin{itemize}
    \item In this part of the course we're assuming that we already know the model, so $\pi(y)$ and $\ell(x|y)$ are known.
    \item If our data follows the model, then no other algorithm (not even deep learning) can outperform the estimators derived by our models
    \item We do not create a generative mechanism but we "pretend" to know it.
\end{itemize}

To do a practical example, we can consider a weather forecasting system, with input data that is temperature, humidity and pressure and output data that is if tomorrow will be sunny or rainy. The \textit{posterior} function (likelihood function) tells us if given that is sunny or rainy what could be the probabile values of the sensor data. The \textit{prior} function tells us on average if it is rainy or sunny.

Now we're going to calculate $f(y|x)$. We've said that apart from a constant $p(x)$, that is constant with respect to $y$ but a function of $x$, our \textit{posterior} function is:
\[
    f(y|x) \propto \pi(y) \ell(x|y)
\]
We know that:
\[
    \ell(x|y) =  \prod_{i = 1}^{N} \frac{1}{\sqrt{2\pi \sigma^2_w}} e^{\frac{-(x_i-y)^2}{2\sigma_w^2}}
\]
and that
\[
    \pi(y) = \frac{1}{\sqrt{2\pi\sigma_y^2}} e^{-\frac{y^2}{2\sigma_y^2}}
\]
So we can write:
\[
    f(y|x) \propto \pi(y) \ell(x|y) = \frac{1}{\sqrt{2\pi\sigma_y^2}} \frac{1}{(2\pi\sigma_w^2)^{\frac{N}{2}}} e^{-\frac{y^2}{2\sigma_y^2}} e^{-\frac{1}{2\sigma_w^2} \sum_{i=1}^{N} (x_i-y)^2}
\]
We can ignore the constants:
\[
    f(y|x) \propto e^{-\frac{y^2}{2\sigma^2_y} - \frac{1}{2\sigma_w^2} \sum_{i=1}^{N} x_i^2 - \frac{N}{2} \frac{y^2}{2 \sigma_w^2} + \frac{y}{\sigma_w^2} \sum_{i=1}^{N} x_i}
\]
Since the second term of the exponential is costant with respect to $y$ and it can be written as a product, it can be ignored because we're considering the fact that is proportional. After ignoring the constant, we can observe that in the following expression the first term depends on the \textit{prior} function while the second term depends on the \textit{posterior} function.
\[
    f(y|x) \propto e^{-\frac{y^2}{2\sigma^2_y} - \frac{N}{2} \frac{y^2}{2 \sigma_w^2} + \frac{y}{\sigma_w^2} \sum_{i=1}^{N} x_i}
\]
We can factor the terms that depend on $y^2$:
\[
    f(y|x) \propto e^{-\frac{y^2}{2}\left(\frac{1}{\sigma_y^2} + \frac{1}{\frac{\sigma^2_w}{N}}\right) + \frac{y}{\sigma_w^2} \sum_{i=1}^{N} x_i}
\]
and then by defining:
\[
    \frac{1}{\sigma^2} = \frac{1}{\sigma_y^2} + \frac{1}{\frac{\sigma^2_w}{N}}
\]
We get:
\[
    f(y|x) \propto e^{-\frac{1}{2\sigma^2} y^2 + \frac{\sum x_i}{\sigma_w^2} y}
\]
Now we try to complete the square by adding and subtracting a function $g(x)$:
\[
    f(y|x) \propto e^{\frac{-(y-\mu)^2}{2\sigma^2}}
\]
where:
\[
    \mu = \frac{\sigma^2}{\sigma_w^2}\sum_{i=1}^{N} x_i
\]
So we obtained that the posterior mean is $\mu$:
\[
    \hat{Y} = \mathbb{E}\left[ Y | X\right] = \frac{\sigma_y^2}{N\sigma_y^2 + \sigma_w^2} \sum_{i=1}^{N} x_i
\]



